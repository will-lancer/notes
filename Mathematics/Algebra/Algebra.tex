\documentclass[11pt]{article}

% basic packages
\usepackage[margin=1in]{geometry}
\usepackage[pdftex]{graphicx}
\usepackage{amsmath,amssymb,amsthm}
\usepackage{william}
\usepackage{tikz-cd}

% page formatting
\usepackage{fancyhdr}
\pagestyle{fancy}

\renewcommand{\sectionmark}[1]{\markright{\textsf{\arabic{section}. #1}}}
\renewcommand{\subsectionmark}[1]{}
\lhead{\textbf{\thepage} \ \ \nouppercase{\rightmark}}
\chead{}
\rhead{}
\lfoot{}
\cfoot{}
\rfoot{}
\setlength{\headheight}{14pt}

\linespread{1.03} % give a little extra room
\setlength{\parindent}{0.2in} % reduce paragraph indent a bit
\setcounter{secnumdepth}{2} % no numbered subsubsections
\setcounter{tocdepth}{2} % no subsubsections in ToC

\begin{document}

% make title page
\thispagestyle{empty}
\bigskip \
\vspace{0.1cm}

\begin{center}
{\fontsize{22}{22} \selectfont Lecture Notes on}
\vskip 16pt
{\fontsize{36}{36} \selectfont \bf \sffamily Algebra}
\vskip 24pt
{\fontsize{18}{18} \selectfont \rmfamily Will Lancer} 
\vskip 6pt
{\fontsize{14}{14} \selectfont \ttfamily will.m.lancer@gmail.com} 
\vskip 24pt
\end{center}

{\parindent0pt \baselineskip=15.5pt}
\noin
Notes on algebra. Resources used:
\begin{itemize}
    \item Aluffi's \emph{Algebra: Chapter 0}
\end{itemize}

% make table of contents
\newpage
\microtoc
\newpage

% main content
\section{Categorical preliminaries}

\note{fill-in}

\section{Groups}

We review the properties of groups here. First,
two high-brow definitions of a group:
\begin{itemize}
    \item A \vocab{group} is a group object in \cat{Set}.
    \item A \vocab{group} is a one-object groupoid.
\end{itemize}
These aren't bad definitions, but they're more explicitly
useful when they're unpacked a bit.
\begin{definition}
    A \vocab{group} is a set with an associative binary operation
    $m_G \colon G \to G$ that has identites and inverses.
\end{definition}
We can phrase this by saying that a group has two additional unique maps,
$\iota \colon G \to G$ by $g \mapsto g^{-1}$ and $\id \colon G \to G$
by $g \mapsto g$. This is the more categorical and ``morally correct''
way of putting this.
\begin{eexample}
    [Canonical group examples]
    The most canonical group examples are
    \begin{itemize}
        \item Any field with addition, $\ints$, $\reals$, $\complex$, $\mathbb{Q}$,
        etc.. Some of them also work with multiplication.
        \item $\ints/n\ints$: the integers mod $n$ under addition. $(\ints/n\ints)^\times$
        are the units mod $n$.
        \item $D_n$. The isometries of a planar solid.
        \item $S_n$ and $A_n$. The permutations and even permutations of $n$ objects.
        \item Matrices over some field under addition or multiplication (if the operation
        is multiplication then needs invertibility).
        \item Types of functions under addition, e.g. smooth, periodic, etc..
    \end{itemize}
\end{eexample}

Groups form a category, \cat{Grp}. Its maps are \vocab{group homomorphisms},
which are maps $\varphi \colon G \to G'$ that respect the group structure,
$\varphi(ab) = \varphi(a)\varphi(b)$. One way of saying this is that the
following diagram commutes:\\
\note{put diagram}\\
We now investigate what other universal constructions exist in \cat{Grp}. 
\begin{itemize}
    \item \textbf{Products}: the product in \cat{Grp} is the \vocab{direct product
    group}. This means that for all $\varphi_G \colon A \to G$, $\varphi_{H} \colon A \to H$,
    there exists a unique map $\widetilde{\varphi} \colon A \to G \times H$
    making the diagram
    \begin{center}
        \begin{tikzcd}
        & & G \\
        A \arrow[r, dashed, "\widetilde{\varphi}"] \arrow[urr, bend left=15, "\varphi_G"] \arrow[drr, bend right=15, "\varphi_H"'] & G \times H \arrow[ur, "\pi_G"] \arrow[dr, "\pi_H"'] & \\
        & & H
        \end{tikzcd}
    \end{center}
    commute. Concretely, taking the binary operations on $G$
    and $H$ and defining 
    \begin{align*}
        m_G \times m_H \colon (G \times H) & \times (G \times H) \to G \times H\\
        (m_G \times m_H)((g_1, h_1), \, (g_2, h_2)) & \mapsto (m_G((g_1, g_2)), \, m_H((h_1, h_2))).
    \end{align*}
    gives the set $G \times H$ a group structure
    \begin{align*}
        (g_1, h_1) * (g_2, h_2) = (g_1 g_2, h_1 h_2).
    \end{align*}
    \item \textbf{Coproducts}: there is not a generic coproduct in \cat{Grp}.
    \note{talk about free product and \cat{Ab}}
    \item \textbf{Quotients} \note{add}
\end{itemize}


\end{document}