\documentclass[11pt]{article}

% basic packages
\usepackage[margin=1in]{geometry}
\usepackage[pdftex]{graphicx}
\usepackage{amsmath,amssymb,amsthm}
\usepackage{tikz-cd}
\usepackage{william}

% page formatting
\usepackage{fancyhdr}
\pagestyle{fancy}

\renewcommand{\sectionmark}[1]{\markright{\textsf{\arabic{section}. #1}}}
\renewcommand{\subsectionmark}[1]{}
\lhead{\textbf{\thepage} \ \ \nouppercase{\rightmark}}
\chead{}
\rhead{}
\lfoot{}
\cfoot{}
\rfoot{}
\setlength{\headheight}{14pt}

\linespread{1.03} % give a little extra room
\setlength{\parindent}{0.2in} % reduce paragraph indent a bit
\setcounter{secnumdepth}{2} % no numbered subsubsections
\setcounter{tocdepth}{2} % no subsubsections in ToC

\begin{document}

% make title page
\thispagestyle{empty}
\bigskip \
\vspace{0.1cm}

\begin{center}
{\fontsize{22}{22} \selectfont Lecture Notes on}
\vskip 16pt
{\fontsize{36}{36} \selectfont \bf \sffamily Commutative Algebra}
\vskip 24pt
{\fontsize{18}{18} \selectfont \rmfamily Will Lancer} 
\vskip 6pt
{\fontsize{14}{14} \selectfont \ttfamily will.m.lancer@gmail.com} 
\vskip 24pt
\end{center}

{\parindent0pt \baselineskip=15.5pt}
\noin
Notes on commutative algebra. Resources used:
\begin{itemize}
    \item Atiyah and MacDonald's \emph{Introduction to Commutative Algebra}.
    \item Altman and Kleiman's \emph{A Term of Commutative Algebra}.
    \item Aluffi's \emph{Algebra: Chapter 0}
\end{itemize}

% make table of contents
\newpage
\microtoc
\newpage

% main content
\section{Commutative rings}

Just how groups are ``sets decorated with structure'', rings
are ``abelian groups decorated with multiplication''. Formally,
this means that
\begin{definition}
    A \vocab{commutative ring} is an abelian group equipped with a multiplication
    operation that is distributive, associative, and commutative.
\end{definition}
We take every ring to have $1_R$, the multiplicative
identity.
\begin{eexample}
    [Examples of rings]
    Here are some examples of rings:
    \begin{itemize}
        \item Fields are rings with division, i.e. $(\ints, +, \cdot)$,
        $(\reals, +, \cdot)$, $(\mathbb{Q}, +, \cdot)$, $(\complex, +, \cdot)$,
        etc. So every example of a field is an example of a ring.
        \item Polynomial rings over a field $k$ in $n$ variables: $k[x_1, \ldots, x_n]$.
        In fact, by the UMP for polynomials, every ring can be uniquely ``embedded''
        into a polynomial ring in the obvious way. Some examples here are $\reals[x]$,
        $\complex[x]$, and $\ints[x]$.
        \item $\ints/p\ints$ is a ring if $p$ is prime.
        \item Matrix rings, $M_n(R)$, where $R$ is our ring. Concretely, these are
        matrices with entries taken from $R$. One can also take these over a finite
        field, which is cool.
        \item Rings of functions, like $C([0, 1], \reals)$ (the set of real-valued 
        continuous functions on $[0, 1]$)
        \item Some more exotic examples: Boolean rings and the quaternions $\mathbb{H}$.
    \end{itemize}
\end{eexample}

Rings form a category, {\cat{Ring}}; its morphisms are constructed as
to have the following diagram commute (if $A$ and $B$ are rings) for
both addition and multiplication:

\[
\begin{tikzcd}
A \times A \arrow[r, "\phi \times \phi"] \arrow[d, "m_A"'] & B \times B \arrow[d, "m_B"] \\
A \arrow[r, "\phi"] & B
\end{tikzcd}
\]

This guarantees \vocab{ring homomorphisms} preserve the ring
structure, i.e. $\phi(a + b) = \phi(a) + \phi(b)$ and $\phi(ab) = \phi(a)\phi(b)$.
A psychological remark here is that ring morphisms are more
beefy than group morphisms, but that's expected, as rings
are more beefy than groups.

We want to have some sort of natural quotient structure
on a ring. The way we do this is through an \vocab{ideal}
\begin{definition}
    An \vocab{ideal} $\mathfrak{a}$ is a subset of $A$
    that satisfies $A\mathfrak{a} \subseteq \mathfrak{a}$.
\end{definition}
Intuitively, you ``can't move out of $\mathfrak{a}$ with $A$'',
or (calling reference to representation theory) an
ideal is an ``invariant subspace'' of a ring.

We now state a useful and obvious theorem: that
some notion of size is preserved when you take the quotient
\note{finish}

\end{document}