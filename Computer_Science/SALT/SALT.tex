\documentclass[11pt]{article}

% basic packages
\usepackage[margin=1in]{geometry}
\usepackage[pdftex]{graphicx}
\usepackage{amsmath,amssymb,amsthm}
\usepackage{william}

% page formatting
\usepackage{fancyhdr}
\pagestyle{fancy}

\renewcommand{\sectionmark}[1]{\markright{\textsf{\arabic{section}. #1}}}
\renewcommand{\subsectionmark}[1]{}
\lhead{\textbf{\thepage} \ \ \nouppercase{\rightmark}}
\chead{}
\rhead{}
\lfoot{}
\cfoot{}
\rfoot{}
\setlength{\headheight}{14pt}

\linespread{1.03} % give a little extra room
\setlength{\parindent}{0.2in} % reduce paragraph indent a bit
\setcounter{secnumdepth}{2} % no numbered subsubsections
\setcounter{tocdepth}{2} % no subsubsections in ToC

\begin{document}

% make title page
\thispagestyle{empty}
\bigskip \
\vspace{0.1cm}

\begin{center}
{\fontsize{22}{22} \selectfont Lecture Notes on}
\vskip 16pt
{\fontsize{36}{36} \selectfont \bf \sffamily SALT}
\vskip 24pt
{\fontsize{18}{18} \selectfont \rmfamily Will Lancer} 
\vskip 6pt
{\fontsize{14}{14} \selectfont \ttfamily will.m.lancer@gmail.com} 
\vskip 24pt
\end{center}

{\parindent0pt \baselineskip=15.5pt}
\noin
Notes on SALT.

% make table of contents
\newpage
\microtoc
\newpage

% main content

\section{SALT tutorial}

I'll be making notes here in an itemized list, as there shouldn't be
deep theory behind this. These should be roughly in sequential
order, but I'll make comments as well. Note that whenever you
see something like wlancer, you should just replace it with whatever
makes sense for your case.

Note that some of the text goes off of the page; this is because
the \verb|verbatim| environment that lets me write code-looking
material forces that to occur. You can still highlight it like normal
to copy and paste it.

\begin{itemize}
    \item Fun remark: one of the talks that they recommend you look
    at before going through this tutorial was done by Professor Dao.
    It has many grumpy cat photos and memes. Apparently CMS does calibration
    better than us? Or something. According to slide 27 they are
    not making the right decision$\ldots$
    \item You're going to have to run on a GPU shell to get anything
    done. This is accomplished by just changing your ssh login
    to \verb|ssh -Y wlancer@lxplus-gpu.cern.ch|. You can then access
    all of your files like normal from here. You should check to
    see if your node is actually configured with GPUs by running
    the \verb|nvidia-smi| command.
    \item You then have to copy over the files into your directory.
    Note that the files are 17GB, so check your EOS to see if you
    have enough storage (you totally should; I had almost two TB). 
    Check your storage using this command.
    \begin{verbatim}
        eos root://eosuser.cern.ch quota /eos/user/w/wlancer
    \end{verbatim}
    This will tell you how many GB you've used, and how many you have
    available.
    \item Copy over the files using
    \begin{verbatim}
        rsync -vaP /eos/user/u/umami/tutorials/salt/2023/inputs/ 
        /eos/user/${USER:0:1}/$USER/training-samples
    \end{verbatim}
\end{itemize}

\begin{iidea}
    [Mounting a ``SALT-ready singularity image'']
    Let's talk about what a \vocab{SALT-ready singularity image}
    is. I'll go in reverse order. An \vocab{image} is a file
    that stores a container's entire filesystem and metadata.
    It's kind of like putting an entire git repo into a file.
    \vocab{Singularity} is a tool that's popular on
    HPC (high-performance computing) systems. It reads in image
    files (which usually end with a \verb|.sif| suffix; sif
    stands for Singularity image format), and
    runs them, giving you reproducible environments. SALT-ready
    just means that the environment that Singularity unpacks is
    one that you may directly run SALT on, i.e. the image contained
    the necessary files and dependencies for SALT.\\

    \vocab{Mounting} a sif file means ``unpacking'' it. This
    is like having an envelope with ten pieces of paper in
    it (the sif file), and then opening said envelope (to get
    the \verb|/usr|, \verb|/bin|, \verb|/lib|, etc. files within).
    It's taking the sif file, which you can't ``look inside'', and then
    basically turning it into a directory.
\end{iidea}

\begin{itemize}
    \item The path
    \begin{verbatim}
        /cvmfs/unpacked.cern.ch/gitlab-registry.cern.ch
        /atlas-flavor-tagging-tools/algorithms/salt:0-3
    \end{verbatim}
    is the container holding the image tagged ``0-3'', which
    stands for the $0.3$ version of SALT. Run the singularity shell
    command; if you get a prompt like \verb|Singularity>|,
    that means you're in the shell and have succeeded.
    For posterity, here is the command, as you're going to have to
    run it every time you open a terminal.
    \begin{verbatim}
        singularity shell -e --env KRB5CCNAME=$KRB5CCNAME --nv --bind $PWD,/afs,/eos,/tmp,/cvmfs,/run/user \
        /cvmfs/unpacked.cern.ch/gitlab-registry.cern.ch/atlas-flavor-tagging-tools/algorithms/salt:0-3
    \end{verbatim}
\end{itemize}

\subsection{Fork, clone, and install Salt}

Just follow the steps they give. They are relatively
straightforward.

\begin{itemize}
    \item Fork it on Gitlab. Clone it to your computer
    using \verb|git clone [link]|. Run the following
    commands to have the original project be upstream
    from yours, and to switch to the correct version
    of SALT.
    \begin{verbatim}
        cd salt
        git remote add upstream ssh://git@gitlab.cern.ch:7999/atlas-flavor-tagging-tools/algorithms/salt.git
        git checkout 0.3
    \end{verbatim}
    \item Cd into the singularity shell using the previous command. 
    Then run the tests suite from the salt directory by running 
    \verb|pytest --cov=salt tests/|. This will run all of the tests in the 
    test suite, and should take about 30 minutes.
\end{itemize}

\subsection{Install logger}

So, I honestly don't know why we're doing this, but they recommended
it. Just follow the hints they give and you'll be fine. You should add
the key and project name to your bashrc using the following commands:
\begin{verbatim}
    echo 'export COMET_API_KEY="yourAPIkey"' >> ~/.bashrc
    echo 'export COMET_WORKSPACE="yourCometWorkspace"' >> ~/.bashrc
\end{verbatim}

\subsection{Train a subject-based tagger (finally!)}

Ok, now time to actually train something! 

\begin{itemize}
    \item If you copied the files over earlier like they recommended, then
    they should be in your eos under the training-samples directory. My lines
    look like:
    \begin{verbatim}
        train_file: /eos/user/w/wlancer/training-samples/pp_output_train.h5
        val_file: /eos/user/w/wlancer/training-samples/pp_output_val.h5
        norm_dict: /eos/user/w/wlancer/training-samples/norm_dict.yaml
        class_dict: /eos/user/w/wlancer/training-samples/class_dict.yaml
    \end{verbatim}
    \item We want to train now. Note that you should be in the Singularity
    shell before you fit your model. The first thing you should do is increase
    your number of \vocab{workers}. Not really sure what these are, but the
    name is suggestive. Run
    \begin{verbatim}
        cat /proc/cpuinfo | awk '/^processor/{print $3}' | tail -1
    \end{verbatim}
    to figure out how many you have, and then do that number or maybe
    a bit below it. Your system will warn you if you are using too many.
    More workers = faster.
    \item You should also change the max number of epochs to $10$.
    \item \note{we can somehow improve speed using HTCondor. will figure out later}
    \item You can now run the training test using
    \begin{verbatim}
        salt fit --config configs/SubjetXbb.yaml --trainer.fast_dev_run 2
    \end{verbatim}
    If this works out fine (it should run for a single epoch), then
    run
    \begin{verbatim}
        salt fit --config configs/SubjetXbb.yaml --data.move_files_temp /tmp
    \end{verbatim}
    to train it fully. Note that you really should use the
    \verb| --data.move_files_temp /tmp| addition. This
    copys the eos files onto your local drive temporarily,
    and significantly speeds up your training according to the
    \href{https://ftag-salt.docs.cern.ch/training/#slurm-batch}{documentation}.
    This is because SALT doesn't have to query the eos each time
    it wants a file, it can just access them from your local directory.
    \item You may find your results in the \verb|logs| directory.
\end{itemize}

\subsection{Track-based taggers}

Now we train a track-based tagger.

\begin{itemize}
    \item Do the same thing you did for the subjet-based tagger,
    i.e. change the file path names, but now in the \verb|GN2X.yaml|
    file.
    \item Note: just like last time, make sure to change the number of workers/epochs;
    the defaults are likely not what you want.
    \item The command you run to train should be
    \begin{verbatim}
        salt fit --config salt/configs/GN2X.yaml --data.move_files_temp /tmp --trainer.devices=1
    \end{verbatim}
    \item You also have to comment out the \verb|track_origin|
    and \verb|track_vertexing| tasks to avoid errors.
    \item I kept running into memory issues when I did this, and to
    fix it, I reduced the number of workers and the batch size. I found
    that reducing the number of workers frees more memory than reducing the
    batch size, e.g. 10 workers at 200 batch size is more memory efficient
    than 20 workers at 100 batch size.
    \item This training took way longer for me, around 20 minutes per epoch.
    So about three hours.
    \item You can find your results like last time. 
\end{itemize}

\subsection{Tinkering with stuff}

I would just look at the solution in the tutorial after trying 
some stuff out. This is an exploratory phase that can last as long
as you feel like.

\subsection{Evaluation}

\begin{itemize}
    \item In general, you need to train and then evaluate your model on validation
    data. This is that second part.
    \item You just need to run
    \begin{verbatim}
        salt test --config logs/<timestamp>/config.yaml --data.test_file path/to/pp_output_test.h5
    \end{verbatim}
    This uses the test file to run on the model you trained. You will
    need to change the paths to route to your files, so for example
    \begin{verbatim}
        salt test --config salt/logs/SubjetXbb_20250711-T221705/config.yaml --data.test_file /eos/user/w/wlancer/training-samples/pp_output_test.h5
    \end{verbatim}
    \item You can find this data in your logs.
\end{itemize}

\subsection{Making plots}

\begin{itemize}
    \item Copy the plotting script into your working directory
    \begin{verbatim}
        cp /eos/home-u/umami/tutorials/salt/2023/make_plots.py .
    \end{verbatim}
    \item Specify the script to our stuff, i.e.
    \begin{verbatim}
        networks = {
            "SubjetXbb": "salt/logs/SubjetXbb_20250711-T221705_test_pp_output_test.h5/pp_output_test.h5",
            "GN2X": "salt/logs/GN2X_20250722-T220855_test_pp_output_test.h5/pp_output_test.h5"  # Update with your actual path
        }   

        reference = "SubjetXbb"
        test_path = '/eos/user/w/wlancer/training-samples/pp_output_test.h5'
    \end{verbatim}
    \item Use \verb|python make_plots.py| to run the script.
    \item You will get the discriminant and ROC information from the
    plots that you can then use to evaluate the models.
\end{itemize}


\end{document}