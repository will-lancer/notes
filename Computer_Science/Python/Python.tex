\documentclass[11pt]{article}

% basic packages
\usepackage[margin=1in]{geometry}
\usepackage[pdftex]{graphicx}
\usepackage{amsmath,amssymb,amsthm}
\usepackage{william}

% page formatting
\usepackage{fancyhdr}
\pagestyle{fancy}

\renewcommand{\sectionmark}[1]{\markright{\textsf{\arabic{section}. #1}}}
\renewcommand{\subsectionmark}[1]{}
\lhead{\textbf{\thepage} \ \ \nouppercase{\rightmark}}
\chead{}
\rhead{}
\lfoot{}
\cfoot{}
\rfoot{}
\setlength{\headheight}{14pt}

\linespread{1.03} % give a little extra room
\setlength{\parindent}{0.2in} % reduce paragraph indent a bit
\setcounter{secnumdepth}{2} % no numbered subsubsections
\setcounter{tocdepth}{2} % no subsubsections in ToC

\begin{document}

% make title page
\thispagestyle{empty}
\bigskip \
\vspace{0.1cm}

\begin{center}
{\fontsize{22}{22} \selectfont Lecture Notes on}
\vskip 16pt
{\fontsize{36}{36} \selectfont \bf \sffamily Python}
\vskip 24pt
{\fontsize{18}{18} \selectfont \rmfamily Will Lancer} 
\vskip 6pt
{\fontsize{14}{14} \selectfont \ttfamily will.m.lancer@gmail.com} 
\vskip 24pt
\end{center}

{\parindent0pt \baselineskip=15.5pt}
\noin
Notes on Python

% make table of contents
\newpage
\microtoc
\newpage

% main content
\section{NumPy}

\begin{itemize}
    \item \verb|numpy| is a library used to deal with matrices.
    \item There are two broad distinctions here: creating arrays
    and manipulating arrays.
    \item The two most common applications of creating arrays is range-based
    filling and random-based filling. You execute these respectively
    by \verb|np.arange(start, stop + 1)| and \verb|np.random.(...)|.
    \item Next is manipulating arrays. Numpy is built to make this intuitive,
    so your intuition should be enough to manipulate arrays just fine.
\end{itemize}

\section{Pandas}

\begin{itemize}
    \item Pandas is a library that gives us useful data structures to work
    with. The most useful of these is the \vocab{DataFrame}.
    \item DataFrames feel similar to Trees in ROOT. They are grids
    of data, with named columns and numbered rows. A DataFrame takes in an
    array and a list of column names to be created, so
    \begin{verbatim}
        df = pd.DataFrame(data=myData, columns=myColumns)
    \end{verbatim}
    \item You can access specific rows and columns of your DataFrame like
    so. You can also access slices of the DataFrame if you'd like.
    \begin{verbatim}
        # Row
        df.iloc[[rowNum]]
        # Column
        df['column name']
        # Slice
        df[start:stop]
    \end{verbatim}
    \item You may also copy DataFrames, either by reference or by copying.
    Reference is what you think it is (will change both), and copying is just a
    copy. You do them like so.
    \begin{verbatim}
        # By reference
        referenceDf = df
        # By copying
        copyDf = df.copy()
    \end{verbatim}
\end{itemize}

\end{document}