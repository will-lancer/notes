\documentclass[11pt]{article}

% basic packages
\usepackage[margin=1in]{geometry}
\usepackage[pdftex]{graphicx}
\usepackage{amsmath,amssymb,amsthm}
\usepackage{william}
\usepackage{tikz-cd}

% page formatting
\usepackage{fancyhdr}
\pagestyle{fancy}

\renewcommand{\sectionmark}[1]{\markright{\textsf{\arabic{section}. #1}}}
\renewcommand{\subsectionmark}[1]{}
\lhead{\textbf{\thepage} \ \ \nouppercase{\rightmark}}
\chead{}
\rhead{}
\lfoot{}
\cfoot{}
\rfoot{}
\setlength{\headheight}{14pt}

\linespread{1.03} % give a little extra room
\setlength{\parindent}{0.2in} % reduce paragraph indent a bit
\setcounter{secnumdepth}{2} % no numbered subsubsections
\setcounter{tocdepth}{2} % no subsubsections in ToC

\begin{document}

% make title page
\thispagestyle{empty}
\bigskip \
\vspace{0.1cm}

\begin{center}
{\fontsize{22}{22} \selectfont Lecture Notes on}
\vskip 16pt
{\fontsize{36}{36} \selectfont \bf \sffamily QFT}
\vskip 24pt
{\fontsize{18}{18} \selectfont \rmfamily Will Lancer} 
\vskip 6pt
{\fontsize{14}{14} \selectfont \ttfamily will.m.lancer@gmail.com} 
\vskip 24pt
\end{center}

{\parindent0pt \baselineskip=15.5pt}
\noin
Resources used:
\begin{itemize}
    \item Xi Yin's 253ab courses. These lecture notes are the best
    QFT resource on the planet. \note{add more; this is main resource
    for these notes}
    \item Steven Weinberg's two volumes.
    \item Daniel Harlow's notes.
    \item Sidney Coleman's QFT. A volume written by an absolute master
    of the subject. Indeed, Weinberg said that he learned more QFT from Sidney
    than from anyone else. \note{add more when you finish reading}
    \item \emph{FIELDS} by Warren Siegel. \note{i am so conflicted about this
    book bro}
\end{itemize}
We mostly follow Weinberg's notation, except where his notation sucks.

% make table of contents
\newpage
\microtoc
\newpage

% main content

\section{Overview}

\note{overview things here; talk about how single-particle rel
qm is inconsistent}

\section{Relativistic Quantum Mechanics}

We start by recalling the axioms of quantum mechanics\footnote{
    Of course, there are different axioms than this that give
    identical results. ``Axioms'' is really a misnomer; they're
    just assumptions that let us build the rest of the theory.
    There is nothing intrinsically fundamental about this set of
    axioms as opposed to other ones.
}:


\begin{iidea}
    [Axioms of quantum mechanics]
    The axioms of quantum mechanics are:
    \begin{itemize}
        \item Quantum mechanical \vocab{states} are represented by \vocab{rays}
        in a \vocab{Hilbert space}. We will denote Hilbert spaces by $\mathcal{H}$.
        \item \vocab{Observables} are represented by self-adjoint operators on
        our Hilbert space. Observed quantities are the eigenvalues of vectors in $\mathcal{H}$.
        \item Let $\mathcal{R}$ be a ray. If $\mathcal{R}_1, \ldots, \mathcal{R}_n$ are a
        set of mutually orthogonal rays, the probabilty of observing $\mathcal{R}$ in
        $\mathcal{R}_i$ is
        \begin{align*}
            \mathbb{P}(\mathcal{R} \to \mathcal{R}_i) = | \braket{\psi}{\phi_i} |^2,
        \end{align*}
        where $\psi \in \mathcal{R}$ and $\phi_i \in \mathcal{R}_i$.
    \end{itemize}
\end{iidea}

\noin
Remarks:

\begin{itemize}
    \item Recall that rays are equivalence classes of kets up to
    phase and normalization. Since this is an equivalence class,
    WLOG we can choose our representative to be normalized.
    \item Self-adjoint or Hermitian? \note{comment on this}
    \item In our third axiom, we make reference to experiment.
    How do we physically represent an experiment on our Hilbert space?
    \note{good question}
\end{itemize}

The most important kinds of operators are operators that realize
\emph{symmetry transformations} on our Hilbert space.

\note{finish}

\subsection{Symmetries}

\note{add the normal ones, and then ps
in the next section}

\note{for whatever reason}, this continuity may also be found in the
\emph{representation} of the symmetry.

\subsection{Poincaré symmetry and the little group}
Since we are doing relativistic quantum mechanics, let's
see how the relativistic symmetry group acts on our Hilbert space.
The Poincaré group in mostly-plus is given by
\begin{align*}
    \mathcal{P} = \reals^D \sdprod SO(D - 1, 1).
\end{align*}
Consider some Poincaré transformation $(\Lambda\indices{^\mu_\nu}, a^\mu)$.
By Wigner's theorem, this is represented as a linear unitary operator on
our Hilbert space; the group property is
\begin{align*}
    U(\Lambda, a) U(\bar{\Lambda}, \bar{a}) = U(\Lambda \bar{\Lambda}, \Lambda \bar{a} + a).
\end{align*}
Since the Poincaré group is a Lie group, things are continuous.
Taylor-expanding $U(\Lambda, a)$ gives
\begin{align*}
    U(\Lambda, a) \simeq \delta^{\mu}_{\nu} - i a_\mu \widehat{\mathcal{P}} + \frac{i}{2} \omega_{\mu \nu} \widehat{J}^{\mu \nu}.
\end{align*}
The generators of $U(\Lambda, a)$ are very special: they are the
four-momentum operator and the angular momentum/boost operator.
Looking at $\widehat{J}^{\mu \nu}$ gives
\begin{align*}
    \widehat{J}^{\mu \nu} = \begin{cases}
        \note{put the boost angular momentum ops here}
    \end{cases}
\end{align*}
\note{talk about general properties of $U(\Lambda)$ here first}
We would like to see how $U(\Lambda, a)$ acts on our states now.
Choose our states to be labeled as $\ket{\bfk, \sigma}$, where
$\bfk$ is our spatial momentum and $\sigma$ is some set of internal
degrees of freedom (which have not yet been specified). Note that
we may well use $\ket{k, \sigma}$ instead of $\ket{\bfk, \sigma}$,
but I like $\ket{\bfk, \sigma}$ more because it shows that these
particles are all on-shell (as $k^0 = \sqrt{|\bfk|^2 + m^2}$ here).
This basis diagonalizes translations, i.e.
\begin{align*}
    U(0, a) \ket{\bfk, \sigma} = e^{- i \widehat{\mathcal{P}} a} \ket{\bfk, \sigma} = e^{-i k a} \ket{\bfk, \sigma}.
\end{align*}
Let us now study how Lorentz transformations act on these states.
Define an arbitrary ket state as
\begin{align*}
    \ket{\bfk, \sigma} \coloneqq N(\bfk) U(L(\bfk)) \ket{\bfk_R, \sigma},
\end{align*}
where $\bfk_R$ is a given \vocab{reference momentum}\footnote{This will
be specific to the kinds of particles you are talking about, e.g.
$k_R = (E, 0, 0, E)$ is a reference momentum for the massless little
group in $D = 4$.}, $L(\bfk)$ is just some Lorentz transformation 
that takes $\bfk_R \xrightarrow{L(\bfk)} \bfk$, and $N(\bfk)$
a normalization factor to be determined. Consider some $U(\Lambda)$
for arbitrary $\Lambda$. We have
\begin{align*}
    U(\Lambda) \ket{\bfk, \sigma} & = N(\bfk) U(\Lambda) U(L(\bfk)) \ket{\bfk_R, \sigma}\\
    & = N(\bfk) N(\boldsymbol{\Lambda k})^{-1} U(L(\boldsymbol{\Lambda k})) U(\underbrace{L(\boldsymbol{\Lambda k})^{-1} \Lambda L(\bfk))}_{W}) \ket{\bfk_R, \sigma}.
\end{align*}
Look at $W$. The following diagram commutes:
\[
    \begin{tikzcd}[column sep=large, row sep=large]
    \bfk_R \arrow[r, "\,L(\bfk)\,"] \arrow[rrd, "\,W\,"'] & \bfk \arrow[r, "\,\Lambda\,"] & \boldsymbol{\Lambda k} \arrow[d, "\,L(\boldsymbol{\Lambda k})^{-1}\,"] \\
    & & \bfk_R
    \end{tikzcd}
\]
Thus $W$ fixes $\bfk_R$. The set of $W$'s forms a group called the
\vocab{little group} for our specific case (e.g. massless $D = 4$,
massive $D = 3$, etc.). Thus $U(W)$ must be a linear combination of
$\ket{\bfk_R, \sigma'}$ vectors, which we can write as
\begin{align*}
    U(W) \ket{\bfk_R, \sigma} = \sum_{\sigma'} D_{\sigma \sigma'}(W) \ket{\bfk_R, \sigma'}.
\end{align*}
Notice that $D(W_1 W_2) = D(W_2) D(W_1)$, as we can see from the definition.
We now determine the normalization from orthogonality. Consider \note{finish}
Putting this all together, we have that
\begin{align*}
    \boxed{U(\Lambda)\ket{\bfk, \sigma} = \sqrt{\frac{\omega_{\boldsymbol{\Lambda k}}}{\omega_{\bfk}}}\sum_{\sigma \sigma'} D_{\sigma \sigma'}(W(\bfk)) \ket{\boldsymbol{\Lambda \bfk}, \sigma'}.}
\end{align*}
\note{why does this hold for multiparticle states? I think because they
are diff vectors, so doesn't matter; can use raising and lowering ops i think}
\begin{align*}
    \boxed{U(\Lambda, a)\ket{\bfk_1, \sigma_1; \bfk_2, \sigma_2; \ldots ; \bfk_n, \sigma_n}
    = \prod_{i = 1}^n e^{-i k_i a}\sqrt{\frac{\omega_{\boldsymbol{\Lambda k}_i}}{\omega_{\bfk_i}}} \sum_{\sigma_i \sigma_i'} D_{\sigma_i \sigma_i'}(W(\bfk_i)) \ket{\boldsymbol{\Lambda \bfk}_1', \sigma_1'; \boldsymbol{\Lambda \bfk}_2, \sigma_2'; \ldots, \boldsymbol{\Lambda \bfk}_n, \sigma_n'}}
\end{align*}
This will be useful when we talk about scattering later.

We now take a closer look at the $D$ matrices. \note{finish}

\subsection{Lagrangian QM}

Now we'll get into Lagrangian QM, which naturally introduces
the path integral and accompanying ideas (regularization, renormalization,
etc.).

\begin{iidea}
    [The path integral]
    Let $\bra{q_f}$ and $\ket{q_i}$ respectively be final
    and initial position eigenstates. Call the time-translation
    operator for a time $T$, $U(T)$. \note{finish}
\end{iidea}

\note{add lagrangian qm; reg, renorm, path integral derivation;
add exercises from yin and weinberg here too}

\section{Classical Field Theory}

\section{Spin-0 QFTs}

\section{Scattering}

We now get into scattering theory. This is probably the most
important part of QFT, so we'll go relatively deep into it.

\subsection{The basics}

Composite particles decay, and fundamental particles




\subsection{Symmetries of the $S$-matrix}

\subsection{The LSZ reduction}

\begin{reemark}
    [On our derivation]
    For the LSZ reduction, we follow Haag and Ruelle by way of Yin.
    You may see his lecture notes on this from 253a linked on the
    first page of these notes.

    We will purposefully state many approximate results. One can get true equalities
    here, but the derivation is much more obtuse, so we will stick to our wave
    packets. Also, everything is basically the same anyways, so I don't care.
\end{reemark}

The goal of the LSZ reduction is to write our in and out
states in terms of our field operators, thus giving us a
connection between the $S$-matrix and perturbation theory.
A priori, there is no connection here to speak about---indeed, 
the LSZ reduction is \textbf{the most important
result in all of quantum field theory}.

We begin by defining a ``smeared field operator'' via
\begin{align*}
    \widehat{\phi}_f \coloneqq \int d^D x \, f(x) \widehat{\phi}(x).
\end{align*}
We assume our smearing function's Fourier transform is supported 
near the mass shell:
\begin{align*}
    \tilde{f}(k) \neq 0 \xleftrightarrow{\: \: \: \: \approx \: \: \: \:} k^2 + m^2 = 0.
\end{align*}
How do we talk about ``moving fields''? This is something we
would like, as our in and out states are assumed to be asymptotically
free; you can't get to an asympototic if you can't move anywhere. 
Consider the following transformation on $f$:
\begin{align*}
    f(k) \mapsto f^{(T)}(k) \text{ by } f^{(T)}(k) = e^{i(k^0 - \omega_\bfk)T} f(k).
\end{align*}
How does this transformation act on our field operators?
\note{add wave equation part here}
Furthermore, the support of $\tilde{f}$ kills the
multi-particle contribution of $\widehat{\phi}_{f} \ket{\Omega}$.
Recall
\begin{align*}
    \widehat{\phi}(x) \ket{\Omega} & = \int \dbar^{D - 1} \bfk \, e^{-ikx}\, \mathcal{Z}^{\rm eff}_{\bfk} \ket{\bfk}
    + \int_{\rm m.p.} d\alpha \, e^{-ip_\alpha x} \, \mathcal{Z}^{\rm eff}_{\alpha} \ket{\alpha}
\end{align*}
Applying the definition of $\widehat{\phi}_f$ and taking the inverse 
Fourier transform of $f$ gives
\begin{align*}
    \widehat{\phi}_f \ket{\Omega} & = \int d^D x \, \int \dbar^{D - 1} \bfk \, e^{-ikx}\, \mathcal{Z}^{\rm eff}_{\bfk} \ket{\bfk}
    + \int d^D x \int_{\rm m.p.} d\alpha \, e^{-ip_\alpha x} \, \mathcal{Z}^{\rm eff}_{\alpha} \ket{\alpha}\\
    & = \int \dbar^{D - 1} \bfk \, \tilde{f}(\bfk, \omega_\bfk)\, \mathcal{Z}^{\rm eff}_{\bfk} \ket{\bfk} + 0
\end{align*}
\note{explain why T transformation doesn't do anything}Thus,
\begin{align*}
    \boxed{\widehat{\phi}_f \ket{\Omega} \approx \widehat{\phi}_{f^{(T)}} \ket{\Omega}.}
\end{align*}
The intuition behind this is that ``there are a lot more operators than
states''. Thus we may define our states as
\begin{align*}
    \ket{\phi_{f_1}, \ldots, \phi_{f_n}} \coloneqq \widehat{\phi}_{f_n} \cdots \widehat{\phi}_{f_1} \ket{\Omega}.
\end{align*}
Thus, by taking inner products of our states, we may get the $S$-matrix
in terms of field operators acting on the vaccuum. We have \note{add $T$
variables on all of these guys below}

\subsection{Analyticity properties of the $S$-matrix}

\note{my goat xi yin}

\note{unitarity bounds too?}

\section{Spin-$1/2$ QFTs}

\section{Spin-$1$ QFTs}

\section{Spin-$3/2$ QFTs}

\section{Spin-$2$ QFTs}

\note{maybe I should find a better organization scheme for this;
this is nice tho; probably will reorganize after I finish learning
QFT II and put all of the content in here or smth}


\end{document}
