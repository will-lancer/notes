\documentclass[11pt]{article}

% basic packages
\usepackage[margin=1in]{geometry}
\usepackage[pdftex]{graphicx}
\usepackage{amsmath,amssymb,amsthm}
\usepackage{william}
\usepackage{tikz-cd}

% page formatting
\usepackage{fancyhdr}
\pagestyle{fancy}

\renewcommand{\sectionmark}[1]{\markright{\textsf{\arabic{section}. #1}}}
\renewcommand{\subsectionmark}[1]{}
\lhead{\textbf{\thepage} \ \ \nouppercase{\rightmark}}
\chead{}
\rhead{}
\lfoot{}
\cfoot{}
\rfoot{}
\setlength{\headheight}{14pt}

\linespread{1.03} % give a little extra room
\setlength{\parindent}{0.2in} % reduce paragraph indent a bit
\setcounter{secnumdepth}{2} % no numbered subsubsections
\setcounter{tocdepth}{2} % no subsubsections in ToC

\begin{document}

% make title page
\thispagestyle{empty}
\bigskip \
\vspace{0.1cm}

\begin{center}
{\fontsize{22}{22} \selectfont Lecture Notes on}
\vskip 16pt
{\fontsize{36}{36} \selectfont \bf \sffamily Groups and Representations}
\vskip 24pt
{\fontsize{18}{18} \selectfont \rmfamily Will Lancer} 
\vskip 6pt
{\fontsize{14}{14} \selectfont \ttfamily will.m.lancer@gmail.com} 
\vskip 24pt
\end{center}

{\parindent0pt \baselineskip=15.5pt}
\noin
Notes on group and representations, as used in physics. Resources used:
\begin{itemize}
    \item Peter van Nieuwenhuizen's PHY 680 course at SBU,
    and the lecture notes from the course.
    \item Andre Lukas' lecture notes on groups and representations.
    \item Fulton and Harris' \emph{Representation Theory: A First Course}.
    \item Parts of Xi Yin's 253ab courses.
\end{itemize}
% make table of contents
\newpage
\microtoc
\newpage

% main content

\section{Finite Groups and Representations}

We begin by collecting the most important facts about the representation
theory of finite groups. At the end, we touch on Majorana spinors and the normal
modes of atomic molecules. For general background, see the notes on
algebra \href{https://github.com/will-lancer/notes/tree/main/Mathematics/Algebra}{here}.

\begin{eexample}
    [The most important finite groups for physics]
    Of course, the most important groups for physics are (semi-simple)
    Lie groups, but there are a few important finite ones as well:
    \begin{itemize}
        \item $S_n$. This group is good for talking about permutations
        of objects, and is necessary for talking about group actions.
        \item $A_n$. See above, but sometimes we only want even permutations.
        \item $D_n$. Because $D_n$ is, by definition, the group of isometries
        of a planar $n$-point object, this is useful to look at when studing
        planar objects. We can also think about certain continuous groups
        as limits of $D_n$, e.g. emergent $O(N)$ symmetry in spin systems.
        \item $\ints/n\ints$. These groups are useful for a number of things.
        $\ints/2\ints$ is parity, and modding out by it also produces quite
        a few useful Lie groups (e.g. \note{give examples here}). It also
        catalogues cyclic groups, so it's useful when we want to think about
        those ideas. Additionally, these groups are closely connected to number
        theory when $n = p$ is prime, and there are more and more connections
        between physics and number theory as time goes on$\ldots$
    \end{itemize}
\end{eexample}

\noin
We now get into the most important definitions behind finite
group theory for physics.

\begin{definition}
    Consider $\gamma_g \colon G \to G$ by $\gamma_g(a) = g a g^{-1}$.
    We say $\gamma_g$ is an \vocab{inner automorphism} of $G$, and we
    denote the set of $\gamma_g$'s by $\operatorname{Inn}_{\text{\cat{Grp}}}(G)$.
\end{definition}

\noin
We may talk about the orbit of some $a$ under the elements of 
$\operatorname{Inn}_{\text{\cat{Grp}}}(G)$; this is called the
\vocab{class} of $a$. The practical algorithm for producing classes
is as follows:
\begin{enumerate}
    \item Take some $a \in G$. Consider $g a g^{-1}$ for all $g$
    in $g$.
    \item If $g a g^{-1}$ is not already in your set, add it to the
    set.
    \item Once you are done, take some $b \in G$ not in the set
    and repeat these steps.
\end{enumerate}
This algorithm terminates, as conjugation is an equivalence relation,
so the classes of $G$ partition it.


\begin{definition}
    The \vocab{center} of a group is defined as
    $\mathcal{Z}(G) \coloneqq \{ z \in G \mid zg = gz \: \: \forall g \in G \}$.
\end{definition}

\noin
Intuitively, the center of a group tells you how abelian the group is.
Indeed, we have that $\mathcal{Z}(G) = G \iff G$ is abelian.
We may ask the question of how to take some non-abelian group
and ``turn it into'' an abelian group. The only natural mechanism
for this would be modding out by a subgroup. As it turns out,
the smallest subgroup that makes this possible is

\begin{definition}
    The \vocab{commutator subgroup} $[G, G]$ of $G$ is
    defined as the group containing the elements $ab a^{-1} b^{-1}$
    for all $a, b \in G$.
\end{definition}

\begin{definition}
    The \vocab{abelianization} of $G$ is defined as $G^{\rm ab} = G/[G, G]$.
\end{definition}



\begin{iidea}
    [Most important facts about finite groups]
    Whenever you see a finite group, you want to answer these questions:
    \begin{itemize}
        \item What is the order of the group? I find this question to be 
        psychologically comforting, and it also tells me what the possible 
        orders of the subgroups of $G$ are.
        \item \textbf{What are its classes?} The reason this is useful is because
        characters are class functions, and so there are a number of useful facts we can
        draw from knowing the classes of a group.
        \item What are the orders of its classes? This is less important, but useful
        if one wants to check whether a map $V$ is a representation or not.
        \item \textbf{What is its center?}  This is useful, because if we mod out 
        by $\mathcal{Z}(G)$, the order of this group is the number of one-dimensional 
        representations of $G$ (so it tells us how many representations we can 
        promptly throw away).
        \item \textbf{What is its commutator subgroup?} As we saw above, the commutator
        subgroup is the smallest subgroup of $G$ such that $G/H$ is abelian.
        \note{add more once you recall this}
    \end{itemize}
\end{iidea}

\end{document}